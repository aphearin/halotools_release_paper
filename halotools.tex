\documentclass[usenatbib,usegraphicx,letterpaper]{mn2e}
\usepackage[totalwidth=480pt,totalheight=680pt]{geometry}

\usepackage{amssymb}
\usepackage{epsfig}
\usepackage{amsmath}
\usepackage{color}
\usepackage[dvipsnames]{xcolor}
%\usepackage{hyperref}
\usepackage{yfonts}

\usepackage{epsfig}  \usepackage{graphicx}   \usepackage{rotating}

%------- New commands

\newcommand{\lsim}{\lower0.6ex\vbox{\hbox{$ \buildrel{\textstyle <}\over{\sim}\ $}}}
\newcommand{\gsim}{\lower0.6ex\vbox{\hbox{$ \buildrel{\textstyle >}\over{\sim}\ $}}}
\newcommand{\beq}{\begin{equation}}
\newcommand{\eeq}{\end{equation}}

%------ Journals

\newcommand{\mnras}{Mon. Not. R. Astron. Soc.}
\newcommand{\apjl}{Astrophys. J. Lett.}
\newcommand{\aj}{Astron. J.}
\newcommand{\aap}{Astron. Astrophys.}
\newcommand{\araa}{Ann. Rev. Astron. Astroph.}
\newcommand{\apjs}{Astrophys. J. Suppl. Ser.}
\newcommand{\physrep}{Phys. Rep.}
\newcommand{\jcap}{JCAP}
\newcommand{\prd}{Phys. Rev. D}
\newcommand{\apj}{ApJ}

\newcommand{\wprp}{w_{\mathrm{p}}}
\newcommand{\rp}{r_{\mathrm{p}}}

\bibliographystyle{mn2e}

%Title of paper---------------------------------------------------------


\title[Halotools]
{
High-Precision Modeling of Large-Scale Structure: \\An open source approach with Halotools}

% Authors ------------------------------------------------------


\author[Hearin et al.]
{Andrew P. Hearin$^{1}$, Duncan Campbell$^{2},$ Erik Tollerud$^{2,3}$\newauthor
many others \\
$^1$Yale Center for Astronomy \& Astrophysics, Yale University, New Haven, CT\\
$^2$Department of Astronomy, Yale University, P.O. Box 208101, New Haven, CT\\
$^3$Space Telescope Science Institute, Baltimore, MD 21218, USA}

\date{Today}

\pagerange{\pageref{firstpage}--\pageref{lastpage}} \pubyear{}

\newtheorem{theorem}{Theorem}[section]

\begin{document}

\maketitle
%----------------------------------------------------------------
%%%%%%%%%%%%%%%%%%%%%%%  A B S T R A C T %%%%%%%%%%%%%%%%%%%%%%%%%%%%%%

\begin{abstract}

We  present the first official release of Halotools, a community-driven python package designed to build and test models of the galaxy-halo connection. Halotools provides a modular platform for creating mock universes with a rich variety of models of galaxy evolution, such as the HOD, CLF, abundance matching, assembly biased models, cored/cuspy NFW profiles, velocity bias, and many other model styles and features. The package has an extensive, heavily optimized toolkit to make mock observations on a synthetic galaxy population, including galaxy clustering, galaxy-galaxy lensing, galaxy group identification, RSD multipoles, void statistics, pairwise velocities and others. Halotools is written in a object-oriented style that enables complex models to be built from a set of simple, interchangeable components, including those of your own creation. Halotools has a rigorously maintained automated testing suite and is exhaustively documented on halotools.readthedocs.org, which includes quickstart guides, source code notes and a large collection of worked examples. The documentation effectively serves as an online textbook on how to build empirical models of galaxy formation in python. We conclude this paper by describing how Halotools can be used to analyze existing datasets to obtain robust constraints on star-formation and quenching processes at low- and high-redshift, and by outlining the Halotools program to help prepare for the arrival of Stage IV dark energy experiments.

\end{abstract} 

%---------------------------
\section{Introduction}
\label{section:introduction}
%---------------------------



%------------------------------------------------

%------------------------------------------------
\end{document}
%------------------------------------------------
